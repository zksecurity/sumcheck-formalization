\documentclass[a4paper,11pt]{article}

\usepackage{amsmath,amssymb,amsthm}
\usepackage[colorlinks=true,linkcolor=blue,citecolor=blue,urlcolor=blue]{hyperref}
\usepackage[capitalize,noabbrev]{cleveref}
\usepackage[margin=1in]{geometry}
\usepackage{enumitem}

% Theorem environments
\theoremstyle{definition}
\newtheorem{definition}{Definition}[section]

\theoremstyle{plain}
\newtheorem{theorem}[definition]{Theorem}
\newtheorem{lemma}[definition]{Lemma}
\newtheorem{corollary}[definition]{Corollary}
\newtheorem{proposition}[definition]{Proposition}

\theoremstyle{remark}
\newtheorem{remark}[definition]{Remark}

% Commands
\newcommand{\FF}{\mathbb{F}}
\newcommand{\NN}{\mathbb{N}}
\DeclareMathOperator{\Prob}{Pr}

\title{The Sum-Check Protocol\\[4pt]
  \large A Self-Contained Reference for Formalization}
\author{}
\date{}

\begin{document}
\maketitle

\begin{abstract}
We present a self-contained description of the sum-check protocol, an
interactive proof protocol introduced by Lund, Fortnow, Karloff, and
Nisan~\cite{LFKN92}. Our presentation is intended to serve as a formalization
target, with explicit definitions, theorem statements, and complete proofs.
We state and prove both completeness and soundness. The soundness proof
proceeds by induction on the number of variables and relies on the
Schwartz--Zippel lemma.
\end{abstract}

\tableofcontents

%% ===================================================================
\section{Introduction}
%% ===================================================================

The sum-check protocol is one of the most fundamental building blocks in the
theory of interactive proofs. It was introduced by Lund, Fortnow, Karloff,
and Nisan~\cite{LFKN92} as a key ingredient in their proof that
$\mathsf{IP} = \mathsf{PSPACE}$. The protocol allows a computationally
unbounded prover to convince a probabilistic verifier of the value of a
multivariate sum
\[
  \sum_{\mathbf{x} \in H^n} g(\mathbf{x}),
\]
where $g$ is a low-degree polynomial over a finite field~$\FF$ and
$H \subseteq \FF$ is a finite subset. The verifier's work is proportional to
the number of variables and the degree of $g$, plus a single evaluation of~$g$,
rather than the exponentially many terms in the sum.

The sum-check protocol has found numerous applications beyond its original
complexity-theoretic context. It underlies the GKR protocol for verifiable
computation~\cite{GKR15} and is a core component of modern succinct
non-interactive arguments of knowledge
(SNARKs)~\cite{Thaler22}.

This document provides a self-contained treatment of the sum-check protocol,
including the Schwartz--Zippel lemma on which the soundness proof relies.
The presentation is structured to serve as a reference for formal verification
in proof assistants: every definition is named and numbered, every theorem has
explicitly stated hypotheses and conclusions, and every proof is complete.

%% ===================================================================
\section{Preliminaries}
%% ===================================================================

Throughout this document, $\FF$ denotes a finite field and $H$ denotes a
non-empty finite subset of~$\FF$. We write $|S|$ for the cardinality of a
finite set~$S$.

%% -------------------------------------------------------------------
\subsection{Multivariate Polynomials}
%% -------------------------------------------------------------------

We write $\FF[X_1, \ldots, X_n]$ for the polynomial ring in $n$~indeterminates
over~$\FF$.

\begin{definition}[Individual degree]\label{def:indiv-deg}
Let $g \in \FF[X_1, \ldots, X_n]$. The \emph{individual degree} of~$g$ in the
variable~$X_i$, written $\deg_{X_i}(g)$, is the largest exponent of~$X_i$
appearing in any monomial of~$g$ with non-zero coefficient. By convention,
$\deg_{X_i}(0) = 0$.

We say that $g$ has \emph{individual degree at most~$d$} if
$\deg_{X_i}(g) \leq d$ for every $i \in \{1, \ldots, n\}$.
\end{definition}

\begin{definition}[Total degree]\label{def:total-deg}
The \emph{total degree} of $g \in \FF[X_1, \ldots, X_n]$, written $\deg(g)$,
is the maximum, over all monomials $X_1^{e_1} \cdots X_n^{e_n}$ with non-zero
coefficient, of the sum $e_1 + \cdots + e_n$. By convention, $\deg(0) = 0$.
\end{definition}

\begin{remark}\label{rem:deg-relation}
If $g$ has total degree at most~$d$, then $g$ has individual degree at most~$d$.
The converse does not hold: for example, $X_1^d \cdot X_2^d$ has individual
degree~$d$ but total degree~$2d$.
\end{remark}

%% -------------------------------------------------------------------
\subsection{The Schwartz--Zippel Lemma}
%% -------------------------------------------------------------------

The following lemma, due independently to Schwartz~\cite{Schwartz80} and
Zippel~\cite{Zippel79}, bounds the probability that a non-zero polynomial
evaluates to zero at a uniformly random point. It is the key tool in the
soundness analysis of the sum-check protocol.

\begin{lemma}[Schwartz--Zippel]\label{lem:schwartz-zippel}
Let $g \in \FF[X_1, \ldots, X_n]$ be a non-zero polynomial of total degree at
most~$d$, and let $S \subseteq \FF$ be a finite non-empty subset. Then
\[
  \Prob_{\mathbf{r} \sim S^n}\bigl[g(\mathbf{r}) = 0\bigr]
  \;\leq\; \frac{d}{|S|}\,,
\]
where $\mathbf{r} = (r_1, \ldots, r_n)$ is drawn uniformly at random
from~$S^n$.
\end{lemma}

\begin{proof}
By induction on~$n$.

\medskip\noindent
\textbf{Base case} ($n = 1$). The polynomial $g \in \FF[X]$ is univariate,
non-zero, and has degree at most~$d$. A non-zero univariate polynomial of
degree at most~$d$ over a field has at most~$d$ roots. Since $r_1$ is chosen
uniformly from~$S$, we have
\[
  \Prob[g(r_1) = 0]
  \;\leq\; \frac{d}{|S|}\,.
\]

\medskip\noindent
\textbf{Inductive step} ($n > 1$). Write $g$ as a polynomial in~$X_n$ with
coefficients in $\FF[X_1, \ldots, X_{n-1}]$:
\[
  g(X_1, \ldots, X_n)
  \;=\; \sum_{j=0}^{k} g_j(X_1, \ldots, X_{n-1}) \cdot X_n^{\,j},
\]
where $k = \deg_{X_n}(g) \leq d$ and $g_k \neq 0$. Every monomial of~$g_k$
arises from a monomial of~$g$ by removing the factor~$X_n^k$; hence
$\deg(g_k) \leq d - k$.

Write $\mathbf{r}' = (r_1, \ldots, r_{n-1})$ and condition on whether
$g_k(\mathbf{r}') = 0$:
\begin{align}
  \Prob[g(\mathbf{r}) = 0]
  &= \Prob\bigl[g(\mathbf{r}) = 0 \mid g_k(\mathbf{r}') = 0\bigr]
     \cdot \Prob\bigl[g_k(\mathbf{r}') = 0\bigr] \notag \\
  &\quad + \Prob\bigl[g(\mathbf{r}) = 0 \mid g_k(\mathbf{r}') \neq 0\bigr]
     \cdot \Prob\bigl[g_k(\mathbf{r}') \neq 0\bigr].
     \label{eq:sz-split}
\end{align}

By the inductive hypothesis applied to $g_k \in \FF[X_1, \ldots, X_{n-1}]$
(which is non-zero of total degree at most $d - k$):
\[
  \Prob\bigl[g_k(\mathbf{r}') = 0\bigr]
  \;\leq\; \frac{d - k}{|S|}\,.
\]

When $g_k(\mathbf{r}') \neq 0$, the univariate polynomial
$g(r_1, \ldots, r_{n-1}, X_n) \in \FF[X_n]$ is non-zero of degree exactly~$k$
(its leading coefficient is $g_k(\mathbf{r}') \neq 0$). By the base case:
\[
  \Prob\bigl[g(\mathbf{r}) = 0 \mid g_k(\mathbf{r}') \neq 0\bigr]
  \;\leq\; \frac{k}{|S|}\,.
\]

Substituting into~\eqref{eq:sz-split} and using that each conditional
probability is at most~$1$:
\[
  \Prob[g(\mathbf{r}) = 0]
  \;\leq\; 1 \cdot \frac{d - k}{|S|} + \frac{k}{|S|} \cdot 1
  \;=\; \frac{d}{|S|}\,. \qedhere
\]
\end{proof}

The following corollary is the form most directly used in the sum-check
soundness proof.

\begin{corollary}[Univariate root bound]\label{cor:univariate-roots}
Let $p \in \FF[X]$ be a non-zero univariate polynomial of degree at most~$d$.
Then for $r$ drawn uniformly at random from~$\FF$:
\[
  \Prob_{r \sim \FF}\bigl[p(r) = 0\bigr]
  \;\leq\; \frac{d}{|\FF|}\,.
\]
\end{corollary}

\begin{proof}
This is \cref{lem:schwartz-zippel} with $n = 1$ and $S = \FF$.
\end{proof}

%% ===================================================================
\section{The Sum-Check Protocol}
%% ===================================================================

%% -------------------------------------------------------------------
\subsection{Problem Statement}
%% -------------------------------------------------------------------

\begin{definition}[Multivariate summation]\label{def:mv-sum}
Let $g \in \FF[X_1, \ldots, X_n]$ and let $H \subseteq \FF$ be a finite
non-empty subset. The \emph{multivariate sum of~$g$ over~$H$} is
\[
  \sigma(g, H, n)
  \;=\; \sum_{(x_1, \ldots, x_n) \in H^n} g(x_1, \ldots, x_n).
\]
\end{definition}

The goal of the sum-check protocol is for a prover to convince a verifier
that $\sigma(g, H, n) = C$ for a claimed value $C \in \FF$.

%% -------------------------------------------------------------------
\subsection{Protocol Description}\label{sec:protocol}
%% -------------------------------------------------------------------

Fix the following data throughout:
\begin{itemize}[nosep]
  \item A finite field~$\FF$.
  \item A non-empty finite subset $H \subseteq \FF$.
  \item A positive integer~$n$ (the number of variables).
  \item A polynomial $g \in \FF[X_1, \ldots, X_n]$ with individual degree at
    most~$d$ in each variable.
  \item A claimed value $C \in \FF$.
\end{itemize}

The protocol proceeds in $n$~rounds. Before round~$1$, set $c_0 = C$.

\medskip
\noindent
\textbf{Round~$i$}, for $i = 1, \ldots, n$:
\begin{enumerate}[label=(\alph*),nosep]
  \item \textbf{Prover's message.}
    The prover sends a univariate polynomial $s_i \in \FF[X]$.

  \item \textbf{Verifier's checks.}
    The verifier rejects if either of the following fails:
    \begin{itemize}[nosep]
      \item \emph{Degree check}: $\deg(s_i) \leq d$.
      \item \emph{Sum check}: $\displaystyle\sum_{a \in H} s_i(a) = c_{i-1}$.
    \end{itemize}

  \item \textbf{Verifier's challenge.}
    The verifier samples $r_i \in \FF$ uniformly at random and sets
    $c_i = s_i(r_i)$.
    \begin{itemize}[nosep]
      \item If $i < n$: the verifier sends $r_i$ to the prover.
      \item If $i = n$ (final round): the verifier performs a
        \emph{final check} --- it accepts if and only if
        \[
          s_n(r_n) = g(r_1, \ldots, r_n).
        \]
    \end{itemize}
\end{enumerate}

\begin{remark}[Oracle access]\label{rem:verifier-oracle}
The verifier evaluates~$g$ at a single point $(r_1, \ldots, r_n)$ in the final
round. This is the only evaluation of~$g$ that the verifier performs. In
complexity-theoretic applications, this evaluation is done via an oracle; in
cryptographic applications, it is typically handled by a polynomial commitment
scheme.
\end{remark}

\begin{remark}[Implementing the degree check]\label{rem:degree-check}
The degree check $\deg(s_i) \leq d$ can be enforced by requiring the prover to
send~$s_i$ as a list of at most $d + 1$ coefficients (or, equivalently, as
evaluations at $d + 1$ predetermined points that uniquely determine a
polynomial of degree at most~$d$). In the soundness analysis, this check
ensures that the difference between the prover's polynomial and the ``correct''
one has degree at most~$d$, enabling the application of
\cref{cor:univariate-roots}.
\end{remark}

%% -------------------------------------------------------------------
\subsection{Completeness}\label{sec:completeness}
%% -------------------------------------------------------------------

We first define the honest prover's strategy, then show that it always causes
the verifier to accept when the claimed sum is correct.

\begin{definition}[Honest prover]\label{def:honest-prover}
The \emph{honest prover} sends, in round~$i$ (given previous challenges
$r_1, \ldots, r_{i-1}$), the polynomial
\[
  s_i^*(X)
  \;=\; \sum_{(x_{i+1}, \ldots, x_n) \in H^{n-i}}
    g(r_1, \ldots, r_{i-1},\, X,\, x_{i+1}, \ldots, x_n).
\]
In round~$1$, there are no prior challenges, so
\[
  s_1^*(X) = \sum_{(x_2, \ldots, x_n) \in H^{n-1}} g(X, x_2, \ldots, x_n).
\]
In round~$n$, $s_n^*(X) = g(r_1, \ldots, r_{n-1}, X)$.
\end{definition}

\begin{theorem}[Completeness]\label{thm:completeness}
If $C = \sigma(g, H, n)$ and the prover follows the honest strategy of
\cref{def:honest-prover}, then the verifier accepts with probability~$1$.
\end{theorem}

\begin{proof}
We verify each of the verifier's checks.

\medskip\noindent
\emph{Degree check.}
For each round~$i$, the polynomial $s_i^*(X)$ is obtained from~$g$ by
substituting field elements for all variables except the $i$-th and summing
over~$H$ in the remaining variables. Since $g$ has individual degree at
most~$d$ in~$X_i$, and substitution and summation do not increase the degree in
the surviving variable, $\deg(s_i^*) \leq d$.

\medskip\noindent
\emph{Sum check in round~$1$.}
\begin{align*}
  \sum_{a \in H} s_1^*(a)
  &= \sum_{a \in H}\; \sum_{(x_2, \ldots, x_n) \in H^{n-1}}
     g(a, x_2, \ldots, x_n) \\
  &= \sum_{(x_1, \ldots, x_n) \in H^n} g(x_1, \ldots, x_n)
  \;=\; \sigma(g, H, n) \;=\; C \;=\; c_0.
\end{align*}

\medskip\noindent
\emph{Sum check in round~$i > 1$.}
\begin{align*}
  \sum_{a \in H} s_i^*(a)
  &= \sum_{a \in H}\; \sum_{(x_{i+1}, \ldots, x_n) \in H^{n-i}}
     g(r_1, \ldots, r_{i-1},\, a,\, x_{i+1}, \ldots, x_n) \\
  &= \sum_{(x_i, \ldots, x_n) \in H^{n-i+1}}
     g(r_1, \ldots, r_{i-1},\, x_i,\, x_{i+1}, \ldots, x_n) \\
  &= s_{i-1}^*(r_{i-1}) \;=\; c_{i-1}.
\end{align*}

\medskip\noindent
\emph{Final check.}
\[
  s_n^*(r_n) = g(r_1, \ldots, r_{n-1}, r_n) = g(r_1, \ldots, r_n).
\]

All checks pass, so the verifier accepts.
\end{proof}

%% -------------------------------------------------------------------
\subsection{Soundness}\label{sec:soundness}
%% -------------------------------------------------------------------

We now prove that if $C \neq \sigma(g, H, n)$, then no prover strategy can
make the verifier accept with high probability.

\begin{definition}[Prover strategy]\label{def:prover-strategy}
A \emph{prover strategy} is a collection of functions
$P = (P_1, \ldots, P_n)$, where
\[
  P_i \colon \FF^{i-1} \longrightarrow \FF[X]
\]
maps the previous challenges $(r_1, \ldots, r_{i-1})$ to a univariate
polynomial. We write $s_i = P_i(r_1, \ldots, r_{i-1})$ for the polynomial
sent in round~$i$.
\end{definition}

\begin{remark}\label{rem:prover-deterministic}
It suffices to consider deterministic prover strategies. Any randomized
strategy is a distribution over deterministic strategies, and the acceptance
probability of the mixture is a convex combination of the acceptance
probabilities of the deterministic strategies. Hence there exists a
deterministic strategy that achieves at least the same acceptance probability.
\end{remark}

\begin{definition}[Verifier acceptance]\label{def:acceptance}
Given data $(\FF, H, n, d, g, C)$ and a prover strategy~$P$, the
\emph{verifier accepts} if all of the following hold (where
$r_1, \ldots, r_n \in \FF$ are the verifier's uniformly random challenges):
\begin{enumerate}[label=(\roman*),nosep]
  \item $\deg(s_i) \leq d$ for all $i \in \{1, \ldots, n\}$;
  \item $\displaystyle\sum_{a \in H} s_1(a) = C$;
  \item $\displaystyle\sum_{a \in H} s_i(a) = s_{i-1}(r_{i-1})$
    for all $i \in \{2, \ldots, n\}$;
  \item $s_n(r_n) = g(r_1, \ldots, r_n)$.
\end{enumerate}
\end{definition}

\begin{theorem}[Soundness]\label{thm:soundness}
Let $g \in \FF[X_1, \ldots, X_n]$ have individual degree at most~$d$ in each
variable, and let $C \neq \sigma(g, H, n)$. Then for any prover strategy~$P$:
\[
  \Prob_{r_1, \ldots, r_n \sim \FF}\bigl[\textup{verifier accepts}\bigr]
  \;\leq\; \frac{n \cdot d}{|\FF|}\,.
\]
\end{theorem}

\begin{proof}
We proceed by induction on~$n$. In each case we may assume without loss of
generality that $\deg(s_i) \leq d$ for all~$i$: if any $s_i$ has degree
greater than~$d$, the verifier rejects immediately, and the acceptance
probability is~$0$.

\medskip\noindent
\textbf{Base case} ($n = 1$).
The prover sends $s_1 \in \FF[X]$ with $\deg(s_1) \leq d$.
Define the honest polynomial $s_1^*(X) = g(X)$.
The verifier checks $\sum_{a \in H} s_1(a) = C$.
Since
\[
  \sum_{a \in H} s_1^*(a)
  = \sum_{a \in H} g(a)
  = \sigma(g, H, 1) \neq C
  = \sum_{a \in H} s_1(a),
\]
we have $s_1 \neq s_1^*$. The polynomial $s_1 - s_1^*$ is non-zero of degree
at most~$d$. By \cref{cor:univariate-roots}:
\[
  \Prob_{r_1 \sim \FF}\bigl[s_1(r_1) = s_1^*(r_1)\bigr]
  = \Prob_{r_1 \sim \FF}\bigl[(s_1 - s_1^*)(r_1) = 0\bigr]
  \leq \frac{d}{|\FF|}\,.
\]
Since the verifier accepts only if $s_1(r_1) = g(r_1) = s_1^*(r_1)$, the
acceptance probability is at most $d / |\FF|$.

\medskip\noindent
\textbf{Inductive step} ($n > 1$).
Assume the theorem holds for all instances with $n - 1$ variables.

The prover sends $s_1 \in \FF[X]$ with $\deg(s_1) \leq d$ in round~$1$.
The verifier checks $\sum_{a \in H} s_1(a) = C$.
Define the honest round-$1$ polynomial:
\[
  s_1^*(X)
  \;=\; \sum_{(x_2, \ldots, x_n) \in H^{n-1}} g(X, x_2, \ldots, x_n).
\]
Since $g$ has individual degree at most~$d$ in~$X_1$, we have
$\deg(s_1^*) \leq d$. Moreover:
\[
  \sum_{a \in H} s_1^*(a) = \sigma(g, H, n) \neq C = \sum_{a \in H} s_1(a),
\]
so $s_1 \neq s_1^*$.

The verifier samples $r_1 \sim \FF$ uniformly at random. Define the event
\[
  E \;=\; \bigl\{s_1(r_1) \neq s_1^*(r_1)\bigr\}.
\]
By \cref{cor:univariate-roots} applied to the non-zero polynomial
$s_1 - s_1^*$ of degree at most~$d$:
\begin{equation}\label{eq:neg-E-bound}
  \Prob[\lnot E] \;=\; \Prob\bigl[s_1(r_1) = s_1^*(r_1)\bigr]
  \;\leq\; \frac{d}{|\FF|}\,.
\end{equation}

After round~$1$, the protocol continues with rounds $2, \ldots, n$. These
rounds constitute a sum-check protocol for the $(n-1)$-variate polynomial
\[
  g'(X_2, \ldots, X_n) \;=\; g(r_1, X_2, \ldots, X_n)
\]
with claimed sum $c_1 = s_1(r_1)$. The polynomial~$g'$ has individual degree at
most~$d$ in each of its $n - 1$ variables, and its true sum is
\[
  \sigma(g', H, n-1)
  = \sum_{(x_2, \ldots, x_n) \in H^{n-1}} g(r_1, x_2, \ldots, x_n)
  = s_1^*(r_1).
\]

\textbf{When $E$ occurs:}
$c_1 = s_1(r_1) \neq s_1^*(r_1) = \sigma(g', H, n-1)$, so the claimed sum for
the sub-protocol is incorrect. By the inductive hypothesis applied
to~$g'$:
\[
  \Prob\bigl[\text{verifier accepts rounds $2, \ldots, n$} \mid E\bigr]
  \;\leq\; \frac{(n-1) \cdot d}{|\FF|}\,.
\]

\textbf{When $\lnot E$ occurs:}
The claimed sub-sum happens to be correct, and the verifier may accept. We
bound this contribution by the probability of $\lnot E$ itself.

\medskip
Combining:
\begin{align*}
  \Prob[\text{accept}]
  &= \Prob[\text{accept} \mid E]\,\Prob[E]
     \;+\; \Prob[\text{accept} \mid \lnot E]\,\Prob[\lnot E] \\
  &\leq \Prob[\text{accept} \mid E] + \Prob[\lnot E]
  \tag{since $\Prob[E] \leq 1$ and $\Prob[\text{accept} \mid \lnot E] \leq 1$}
  \\
  &\leq \frac{(n-1) \cdot d}{|\FF|} + \frac{d}{|\FF|}
  \;=\; \frac{n \cdot d}{|\FF|}\,. \qedhere
\end{align*}
\end{proof}

\begin{remark}[Non-uniform degree bounds]\label{rem:general-degrees}
\Cref{thm:soundness} generalizes to the setting where $g$ has individual degree
at most~$d_i$ in variable~$X_i$. In this case, the soundness bound becomes
\[
  \frac{\sum_{i=1}^{n} d_i}{\,|\FF|\,}\,.
\]
The proof is identical, replacing~$d$ by~$d_i$ in round~$i$.
\end{remark}

\begin{remark}[The Boolean hypercube]\label{rem:boolean}
The most common instantiation takes $H = \{0, 1\}$, so that the sum ranges over
the Boolean hypercube $\{0, 1\}^n$. In this case, the sum checks simplify to
$s_i(0) + s_i(1) = c_{i-1}$.
\end{remark}

\begin{remark}[Meaningful soundness]\label{rem:field-size}
The soundness bound $nd/|\FF|$ is only useful when $|\FF| \gg nd$.
In typical applications, $\FF$ is a cryptographically large prime field
(e.g.\ $|\FF| \geq 2^{128}$), so the bound is negligibly small.
\end{remark}

%% ===================================================================
\begin{thebibliography}{99}

\bibitem{LFKN92}
C.~Lund, L.~Fortnow, H.~Karloff, and N.~Nisan.
\newblock Algebraic methods for interactive proof systems.
\newblock \emph{Journal of the ACM}, 39(4):859--868, 1992.

\bibitem{Schwartz80}
J.\,T.~Schwartz.
\newblock Fast probabilistic algorithms for verification of polynomial
  identities.
\newblock \emph{Journal of the ACM}, 27(4):701--717, 1980.

\bibitem{Zippel79}
R.~Zippel.
\newblock Probabilistic algorithms for sparse polynomials.
\newblock In \emph{Symbolic and Algebraic Computation (EUROSAM~'79)},
  volume~72 of \emph{Lecture Notes in Computer Science}, pages 216--226.
  Springer, 1979.

\bibitem{GKR15}
S.~Goldwasser, Y.\,T.~Kalai, and G.\,N.~Rothblum.
\newblock Delegating computation: Interactive proofs for muggles.
\newblock \emph{Journal of the ACM}, 62(4):27:1--27:64, 2015.

\bibitem{Thaler22}
J.~Thaler.
\newblock \emph{Proofs, Arguments, and Zero-Knowledge}.
\newblock Foundations and Trends in Privacy and Security, vol.~4, no.~2--4.
  Now Publishers, 2022.

\end{thebibliography}

\end{document}
